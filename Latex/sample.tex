\documentclass{cpsc413Solutions}
\usepackage{amsmath}

\coursetitle{Design and Analysis of Algorithms}
\courselabel{CPSC 413}
\exercisesheet{Problem Set \#[1]}{}
\student{Name - ID}
\semester{Winter 2020}

\begin{document}

\team{John von Neumann,
Donald Knuth,
Euclid of Alexandria}

\sources{https://math.stackexchange.com/questions/2082706,
Art of Computer Programming Vol 2 p.101-107,
etc,
etc,
etc,
etc}

\begin{problemlist}
\pbitem [Solving World Hunger]
\begin{problem}
\begin{answer}
Your answer here.
\newline
How about some math: $a \equiv b \pmod{n}$.

Or you can display math:
\begin{equation}
\frac{n!}{k!(n-k)!} = \binom{n}{k}
\end{equation}

If you don't like the equation numbers:
\begin{equation*}
\sum_{i=1}^n i = \frac{n(n+1)}{2}
\end{equation*}
\end{answer}
\end{problem}


\pbitem [Desirable properties in crypto software]
\begin{problem}
\begin{answer}
Let's list some:
\begin{itemize}
\item Efficient
\item Secure
\item User friendly
\item \dots
\end{itemize}
\end{answer}
\end{problem}

\end{problemlist}


\end{document}
